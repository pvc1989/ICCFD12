\documentclass[10pt]{article}
\usepackage[a4paper,left=2.54cm,top=2.54cm,right=2.54cm,bottom=2.54cm]{geometry}
\usepackage{fancyhdr}
\setlength{\headsep}{1.cm} % Adjust the space after the header
\usepackage{afterpage}
\usepackage{setspace}
\usepackage{bibspacing}
\usepackage{float}
\singlespacing

%%%% YOU CAN PUT YOUR OWN DEFINITIONS HERE
\newfont{\toto}{msbm10 at 12 pt}
\newfont{\ithd}{cmr9}
\newcommand{\equa}[1]{(\ref{eq:#1})}
\newcommand{\laeq}[1]{\label{eq:#1}}
\newcommand{\figu}[1]{\ref{fig:#1}} 
\newcommand{\lafi}[1]{\label{fig:#1}}
\newcommand{\fmo}{\tilde{U}}
\newcommand{\fve}{\tilde{u}}
\newcommand{\Dt}{\Delta t}

\newcommand{\R}{\mathbb{R}}
\newcommand{\Z}{\mathbb{Z}}
\newcommand{\si}[1]{\rm\scriptscriptstyle{#1}}
%%%% END OF YOUR DEFINITIONS 

\pagestyle{fancyplain}
\renewcommand{\headrulewidth}{0pt}

\usepackage{amsmath,amsthm,amsfonts,amssymb}
\usepackage[pdftex]{graphicx}
\usepackage[T1]{fontenc}

%%%% CONFERENCE HEADER. REPLACE xxxx WITH 4-DIGIT PAPER NUMBER ASSIGNED BY CONFERENCE COMMITTEE.

\rhead{\ithd{\bf ICCFD12-2024-xxxx\\  \   \\}}
\lhead{\ithd{\bf Twelfth International Conference on \\      
Computational Fluid Dynamics (ICCFD12), \\
Kobe, Japan, July 14-19, 2024
}}


\usepackage{titling}
\setlength{\droptitle}{0em}  
\pretitle{\vspace{-4em}\begin{center}\LARGE}
\posttitle{\end{center}\vspace{-1em}}
\preauthor{\begin{center}\large}
\postauthor{\end{center}\vspace{-6em}}


\title{
\bf A Novel Energy-based Artificial Viscosity for Suppressing Numerical Oscillations in Discontinuous Gakerlin and Flux Reconstruction Schemes
}
\author{
Weicheng Pei$^{*}$ and Yu-Xin Ren$^{*}$\\
Corresponding author: weicheng.pei@icloud.com\\
$^{*}$ Department of Engineering Mechanics, Tsinghua University, Beijing 100084, China
}
\date{}

\begin{document}

%%%% TITLE
\maketitle
\afterpage{\fancyhead{}}

%%%% ABSTRACT AND KEYWORDS
%\vskip0.5cm
\centerline{
}
\vskip0.5cm 

%%%% MAIN PART
\section{Introduction}
To construct high-order schemes on unstructured meshes, the discontinuous Gakerlin (DG) method and the flux reconstruction (FR) method are two of the most popular choices in the community of computational fluid dynamics (CFD).
%
Besides the inherent compactness, these schemes are much easier to implement $p$-refinement than their finite difference and finite volume counterparts.
%
However, like other high-order schemes, numerical oscillations would appear near discontinuities or large gradients in the solution given by a DG or FR scheme, if no shock capturing mechanism were incorporated into it.

In this paper, a novel artificial viscosity based on an energy measure of oscillation and its damping rate on a DG or FR element is developed.
%
The oscillation energy, which measures the amplitude of numerical oscillations on a given element, is obtained by evaluating the $L_2$-norm of the difference between the numerical solutions on the element and its neighbors.
%
The damping rate of this energy on an element can be derived under the assumptions of linear flux--gradient relation and constant viscosity distribution.
%
The value of viscosity for suppressing numerical oscillations is obtained by taking the ratio of the oscillation energy with respect to the product of its damping rate and prescribed time step.
%
Such element-wise constant viscosity distribution is reconstructed to be $C_0$ constinuous on element interfaces.

Standard cases for testing shock capturing schemes show that the proposed mechanism is sufficiently large for suppressing numerical oscillations near physical discontinuities, such as shocks and contacts, while keeps negligible in other regions for maintaing the high-orderness of the DG or FR solution.

\section{Problem Statement}
This document allows you to easily include references \cite{book,journalpaper}, equations, figures (see Figure \figu{logo}) or anything else you
desire into a clean and compact environment of \LaTeX.  For example if you'd like to impress a date you can write
the unsteady heat equation as

\begin{eqnarray}
\frac{\partial \mathbf{V}}{\partial t} - \alpha \left( \frac{\partial^2 \mathbf{V}}{\partial x^2} +
       \frac{\partial^2 \mathbf{V}}{\partial y^2} +
       \frac{\partial^2 \mathbf{V}}{\partial z^2} \right)
= 0
\laeq{heat}
\end{eqnarray}
where $x, y, z$ are the space dimensions and $\alpha$ is a parameter.  If you felt inclined you could define $\mathbf{V}$ as
%
$$\mathbf{V} = y^2 z - \text{cos}(0.1 x)$$
%
for a non-exact solution.  Computational fluid dynamics~\cite{paper} can be used to discretize the equations, apply boundary conditions and 
simulation the unsteady nature of the flow.  An innovative method to simulate the heat equation could even be submitted to ICCFD10.

% \begin{figure}[H]
%   \centering
%   \includegraphics[height=2.0in]{exampleFigure.jpg}
%   \caption{This is the logo of ICCFD.}
%   \lafi{logo}
% \end{figure}


\section{Subsection Title Example}

\subsubsection{Sub-subsection Title Example}



%%%% BIBLIOGRAPHY
\bibspacing=\dimen 100
\bibliographystyle{unsrt}
\bibliography{biblio}

\end{document}
